\section{Background}
In this part, we would introduce some natural characters of EVD and historical records of EVD outbreaks. This will help us understand how it spreads and determine appropriate values for the parameters in our models, such as infection rate, recovery rate, etc.

\subsection{Characters of Ebola and its spread}

\par Many factors related to Ebola influence the spread of EVD or influence the construction of our models. These factors are divided into \textbf{Social factors} and \textbf{Biological factors} and they are listed as follows.

\subsubsection{Biological factors}
\begin{itemize}
  \item \textbf{Infectiousness }\\ Infectiousness directly affects the disease's infection rate, which is defined as morbidity rate for uninfected people but exposed to virus carriers. And infectiousness is mainly affected by virulence of the virus and individual's immunity. In most cases, diseases' infection rate is regarded as constant.\\ 
      
  \item \textbf{Lethality }\\ Lethality is how capable a disease is of causing death. It can be described by lethality rate, the ratio of the dead to the total patients. EVD has an average fatality rate of $55\%-80\%$. Moreover, Individuals who was infected and cured won't be infected again in next ten years.
       
  \item \textbf{Incubation}\\ Incubation means that there is a period during which the infected shows no sign or symptom of the disease and is not contagious\cite{CDC}. Therefore, there may be difficulty to segregate people with infectious diseases. EVD's incubation period is $2$ to $21$ days.
      
  \item \textbf{Route of Transmission} \\EVD is mainly spread through direct contact with blood and body fluids of infected ones\cite{CDC}. These routes of transmission are highly connected to public health situation and personal hygiene.
      
  \item \textbf{Environmental Condition}\\ Ebola virus has moderate tolerance to heat, which indicates that EVD maintains a high infectiousness in most of human settlements. However, Ebola virus will be inactivated when exposed to a temperature over $60^{\circ}$  for $60$ minutes. Environment also affects the spread of EVD in aspect of natural reservoir. Though the reservoir remains unknown, it is reasonable to infer that contact with wide animals adds to the probability of getting infected.
      
%  \item 

\end{itemize}


\subsubsection{Social factors}
\begin{itemize}
  \item \textbf{Population}\\In general, a large population means high potential of disease's spread. People in a populous area have a greater frequency of contacting the others than those in a sparsely populated area. According to the route of transmission, it is obvious that the probability of getting infected would be larger.
   
  \item \textbf{Traffic}\\Convinient traffic encourages population mobility, which contributes to the spread of EVD. However, It also encourages freightage, including medicine.
  
  \item \textbf{Medical Level}\\A society will be less affected by EVD if proper measures are taken efficiently and promptly.
   These measures include segregating patients and strengthening the sanitary control of public places. The manufacture of drugs and vaccines aiming at EVD is also an important part.
  
  \item \textbf{Regional Custom}\\Funeral is considered solemn in the African culture. The dead should be cleaned, kissed and touched before buried. This kind of culture facilitates EVD infections.
  
  \item \textbf{Other Social Factors}\\The spread of diseases is also influenced by factors like social development, health situation, individual's living condition, etc. These factors are not considered to simplify our models.
\end{itemize}

\subsection{Information about the last break of Ebola in 2014}

\par In 2014, the largest outbreak of the disease occurred in West Africa, affecting multiple countries mainly including Guinea, Liberia and Sierra Leone.\cite{CDC}Up to February 4th, there have been 22,495 reported confirmed, probable, or suspected cases of EVD worldwide, with 8981 reported deaths. Among these reported cases, 99.84\% of the cases (22,460) are reported in Guinea, Liberia or Sierra Leone, thus showing a strong regional characteristic and suggesting that we should focus our study mainly within these countries.